% !TEX encoding = UTF-8 Unicode
\documentclass[11pt, a4paper]{article}

\usepackage[T1]{fontenc}
\usepackage[utf8]{inputenc}
\usepackage{lmodern}

\usepackage[top = 50mm, headheight=62pt]{geometry}
\usepackage{titleps}
\usepackage{tabularx, booktabs, multirow}
\usepackage{graphicx, adjustbox, microtype}
\usepackage{lipsum}

\usepackage{amsmath}
\usepackage[linesnumbered,ruled]{algorithm2e}
\usepackage[noend]{algpseudocode}

\makeatletter
\def\BState{\State\hskip-\ALG@thistlm}
\makeatother

\usepackage{multicol}
\usepackage{color}
\usepackage{comment}

\newcommand\doublePoint{::}

\nonstopmode

\usepackage[most]{tcolorbox}

\newtcblisting[auto counter]{texCode}[2][]{
     sharp corners,
     keywordstyle=\color[rgb]{0,0,1},
     fonttitle = \bfseries,
     colframe = gray,
     listing only,
     listing options = {basicstyle = \ttfamily, language = TeX},
     title = Code Source LaTeX - \thetcbcounter : #2, #1
}

\newtcblisting[auto counter]{bashCode}[2][]{
     sharp corners,
     keywordstyle=\color[rgb]{0,0,1},
     fonttitle = \bfseries,
     colframe = gray,
     listing only,
     listing options = {basicstyle = \ttfamily, language = Bash},
     title = Code Source bash - \thetcbcounter : #2, #1
}

\newtcblisting[auto counter]{cppCode}[2][]{
     keywordstyle=\color[rgb]{0,0,1},
     sharp corners,
     fonttitle = \bfseries,
     colframe = gray,
     listing only,
     listing options = {basicstyle = \ttfamily, language = C++},
     title = Code Source C++ - \thetcbcounter : #2, #1
}

\newtcblisting[auto counter]{pytCode}[2][]{
     keywordstyle=\color[rgb]{0,0,1},
     sharp corners,
     fonttitle = \bfseries,
     colframe = gray,
     listing only,
     listing options = {basicstyle = \ttfamily, language = Python},
     title = Code Source Python - \thetcbcounter: #2, #1
}

\newtcblisting[auto counter]{javaCode}[2][]{
     sharp corners,
     keywordstyle=\color[rgb]{0,0,1},
     fonttitle = \bfseries,
     colframe = gray,
     listing only,
     listing options = {basicstyle = \ttfamily, language = Java},
     title = Code Source Java - \thetcbcounter: #2, #1
}

\lstset{%
     numbersep=3mm,
		 numbers=left,
		 numberstyle=\tiny,
		 frame=single,
		 framexleftmargin=6mm,
		 xleftmargin=6mm,
		 literate=%
		  	{doublePoint}{::}2 %
				{code}{CODE}4 %
				{end}{END}3 %
				{endl}{endl}4 %
				{image}{IMAGE}5 %
				{chapter}{CHAPTER}7 %
				% {sub}{SUB_SECTION}9 %
				% {subsub}{SUB_SUB_SECTION}14 %
}

\setlength{\columnseprule}{1pt}
\setlength{\columnsep}{0.5cm}
\def\columnseprulecolor{\color{black}}

\makeatletter
     \renewcommand{\thesection}{\Roman{section}.}
     \renewcommand{\thesubsection}{\Roman{section}.\arabic{subsection}.}
     \renewcommand{\thesubsubsection}{\Roman{section}.\arabic{subsection}.\alph{subsubsection}}
\makeatother

\makeatletter
     \renewcommand{\section}{\@startsection{section}{1}{\z@}%
          {-3.5ex \@plus -1ex \@minus -.2ex}%
          {2.3ex \@plus .2ex}%
          {\reset@font\Large\bfseries	}}
     \renewcommand{\subsection}{\@startsection{subsection}{1}{\z@}%
          {-3.5ex \@plus -1ex \@minus -.2ex}%
          {2.3ex \@plus .2ex}%
          {\reset@font\large\bfseries}}
     \renewcommand{\subsubsection}{\@startsection{subsubsection}{1}{\z@}%
          {-3.5ex \@plus -1ex \@minus -.2ex}%
          {2.3ex \@plus .2ex}%
          {\reset@font\large\bfseries}}
\makeatother

\usepackage{lastpage}

\newpagestyle{style}{
\sethead{}{%
     \begin{tabularx}{\linewidth}[b]{@{}l>{\raggedleft\arraybackslash}X@{}}
          \smash{\raisebox{-0.7\height}{}}& \today \\
          & \huge \bfseries Titre du cours \\
          & Cours particulier n°1 \\ 
          \addlinespace
          \midrule[0.4pt]
     \end{tabularx}}{}
\setfoot{}{\thepage}{}
}

\usepackage{tocstyle}
\usepackage[nottoc, notlof, notlot]{tocbibind}

\usetocstyle{standard}
\usepackage{mathtools}
\usepackage{amssymb}

\usepackage[french]{babel}

\pagestyle{style}

\begin{document}
\section{Exercices}\subsection{Vraies ou fausses}Déterminer parmi les propositions suivantes lesquelles sont vraies :
\begin{itemize}
\item 136 est un multiple de 17 et 2 divise 167.
\item 136 est un multiple de 17 ou 2 divise 167.
\item $\exists x\in \mathbb R,\ (x+1=0\ \textrm{ et }x+2=0)$.
\item $(\exists x\in\mathbb R,\ x+1=0)\textrm{ et }(\exists x\in\mathbb R,\ x+2=0)$.
\item $\forall x\in\mathbb R,\ (x+1\neq 0\textrm{ ou }x+2\neq 0)$.
\item $\exists x\in\mathbb R^*,\ \forall y\in\mathbb R^*,\ \forall z\in\mathbb R^*,\ z-xy=0$;
\item $\forall y\in\mathbb R^*,\exists x\in\mathbb R^*,\ \forall z\in\mathbb R^*,\ z-xy=0$;
\item $\forall y\in\mathbb R^*,\forall z\in\mathbb R^*,\ \exists x\in\mathbb R^*,\ z-xy=0$;
\item $\exists a\in\mathbb R,\ \forall \veps&gt;0,\ |a|&lt;\veps$;
\item $\forall \veps&gt;0,\ \exists a\in\mathbb R,\ |a|&lt;\veps$.
\end{itemize}\newpage\section{Indicators}\section{Vraies ou fausses}pas d'indication :(\newpage\section{Corriges}\section{Vraies ou fausses}\begin{itemize}
\item Cette propositions est fausse, car 2 ne divise pas 167.
\item Cette proposition est vraie, car 136 est un multiple de 17.
\item Cette proposition est fausse, car $x$ devrait être simultanément égal à -1 et à -2.
\item Cette proposition est vraie car $(\exists x\in\mathbb R,\ x+1=0)$ est vraie (il suffit de prendre $x=-1$) et de la même façon $(\exists x\in\mathbb R,\ x+2=0)$ est vraie (il suffit de prendre $x=-2$).
\item Cette proposition est vraie, par exemple car il s'agit de la négation de la proposition 3, qui est fausse.
\item Cette assertion est fausse. Si on considère $x$ n'importe quel réel non nul, alors le choix de $y=1$ et de $z=2x$
fait que $z$ est différent de $xy$.
\item Cette assertion est fausse. Prenons n'importe quel $y$ dans $\mathbb R^*$. 
On voudrait trouver $x$ dans $\mathbb R^*$ tel que, pour tout $z$ dans $\mathbb R^*$, on ait $z=xy$. 
Bien sûr, ce n'est pas possible, car le $x$ que l'on choisit devrait convenir à toute valeur de $z$,
ce qui n'est pas possible car il suffit de considérer un $z$ différent de $xy$.
\item Cette assertion est vraie, car on peut choisir $x$ une fois $y$ et $z$ fixés. On choisit alors $x=z/y$.
\item L'assertion est vraie, il suffit de prendre $a=0$ (convient pour toute valeur de $\veps&gt;0$).
\item Cette assertion est "évidemment" vraie car elle est plus faible que la précédente (on peut choisir
$a$ après $\veps&gt;0$).
\end{itemize}

\end{document}
