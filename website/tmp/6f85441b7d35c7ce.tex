% !TEX encoding = UTF-8 Unicode
\documentclass[11pt, a4paper]{article}

\usepackage[T1]{fontenc}
\usepackage[utf8]{inputenc}
\usepackage{lmodern}

\usepackage[top = 50mm, headheight=62pt]{geometry}
\usepackage{titleps}
\usepackage{tabularx, booktabs, multirow}
\usepackage{graphicx, adjustbox, microtype}
\usepackage{lipsum}

\usepackage{amsmath}
\usepackage[linesnumbered,ruled]{algorithm2e}
\usepackage[noend]{algpseudocode}

\makeatletter
\def\BState{\State\hskip-\ALG@thistlm}
\makeatother

\usepackage{multicol}
\usepackage{color}
\usepackage{comment}

\newcommand\doublePoint{::}

\nonstopmode

\usepackage[most]{tcolorbox}

\newtcblisting[auto counter]{texCode}[2][]{
     sharp corners,
     keywordstyle=\color[rgb]{0,0,1},
     fonttitle = \bfseries,
     colframe = gray,
     listing only,
     listing options = {basicstyle = \ttfamily, language = TeX},
     title = Code Source LaTeX - \thetcbcounter : #2, #1
}

\newtcblisting[auto counter]{bashCode}[2][]{
     sharp corners,
     keywordstyle=\color[rgb]{0,0,1},
     fonttitle = \bfseries,
     colframe = gray,
     listing only,
     listing options = {basicstyle = \ttfamily, language = Bash},
     title = Code Source bash - \thetcbcounter : #2, #1
}

\newtcblisting[auto counter]{cppCode}[2][]{
     keywordstyle=\color[rgb]{0,0,1},
     sharp corners,
     fonttitle = \bfseries,
     colframe = gray,
     listing only,
     listing options = {basicstyle = \ttfamily, language = C++},
     title = Code Source C++ - \thetcbcounter : #2, #1
}

\newtcblisting[auto counter]{pytCode}[2][]{
     keywordstyle=\color[rgb]{0,0,1},
     sharp corners,
     fonttitle = \bfseries,
     colframe = gray,
     listing only,
     listing options = {basicstyle = \ttfamily, language = Python},
     title = Code Source Python - \thetcbcounter: #2, #1
}

\newtcblisting[auto counter]{javaCode}[2][]{
     sharp corners,
     keywordstyle=\color[rgb]{0,0,1},
     fonttitle = \bfseries,
     colframe = gray,
     listing only,
     listing options = {basicstyle = \ttfamily, language = Java},
     title = Code Source Java - \thetcbcounter: #2, #1
}

\lstset{%
     numbersep=3mm,
		 numbers=left,
		 numberstyle=\tiny,
		 frame=single,
		 framexleftmargin=6mm,
		 xleftmargin=6mm,
		 literate=%
		  	{doublePoint}{::}2 %
				{code}{CODE}4 %
				{end}{END}3 %
				{endl}{endl}4 %
				{image}{IMAGE}5 %
				{chapter}{CHAPTER}7 %
				% {sub}{SUB_SECTION}9 %
				% {subsub}{SUB_SUB_SECTION}14 %
}

\setlength{\columnseprule}{1pt}
\setlength{\columnsep}{0.5cm}
\def\columnseprulecolor{\color{black}}

\makeatletter
     \renewcommand{\thesection}{\Roman{section}.}
     \renewcommand{\thesubsection}{\Roman{section}.\arabic{subsection}.}
     \renewcommand{\thesubsubsection}{\Roman{section}.\arabic{subsection}.\alph{subsubsection}}
\makeatother

\makeatletter
     \renewcommand{\section}{\@startsection{section}{1}{\z@}%
          {-3.5ex \@plus -1ex \@minus -.2ex}%
          {2.3ex \@plus .2ex}%
          {\reset@font\Large\bfseries	}}
     \renewcommand{\subsection}{\@startsection{subsection}{1}{\z@}%
          {-3.5ex \@plus -1ex \@minus -.2ex}%
          {2.3ex \@plus .2ex}%
          {\reset@font\large\bfseries}}
     \renewcommand{\subsubsection}{\@startsection{subsubsection}{1}{\z@}%
          {-3.5ex \@plus -1ex \@minus -.2ex}%
          {2.3ex \@plus .2ex}%
          {\reset@font\large\bfseries}}
\makeatother

\usepackage{lastpage}

\newpagestyle{style}{
\sethead{}{%
     \begin{tabularx}{\linewidth}[b]{@{}l>{\raggedleft\arraybackslash}X@{}}
          \smash{\raisebox{-0.7\height}{}}& \today \\
          & \huge \bfseries Khôlles \\
          & kholles generator by planchon.io \\ 
          \addlinespace
          \midrule[0.4pt]
     \end{tabularx}}{}
\setfoot{}{\thepage}{}
}

\usepackage{tocstyle}
\usepackage[nottoc, notlof, notlot]{tocbibind}

\usetocstyle{standard}
\usepackage{mathtools}
\usepackage{amssymb}

\usepackage[french]{babel}

\pagestyle{style}

\begin{document}
\section{Exercices}\subsection{Du texte aux quantificateurs}Soit  $f:\mathbb R\to\mathbb R$ une fonction. 
Exprimer à l'aide de quantificateurs les assertions suivantes :
\begin{itemize}
\item $f$ est constante;
\item $f$ n'est pas constante;
\item $f$ s'annule;
\item $f$ est périodique.
\end{itemize}\subsection{QCM}Pour chaque question, une seule réponse est juste. Laquelle?
\begin{itemize}
\item La somme $\sum_{k=0}^n 2$
$$\mathbf a.\textrm{ n'a pas de sens}\ \ \mathbf b. \textrm{ vaut }2(n+1)\ \ \mathbf c.\ \textrm{vaut }2n.$$
\item La somme $\sum_{p=0}^{2n+1}(-1)^p$ est égale à 
$$\mathbf a.\ 1\ \ \mathbf b.\ -1\ \ \mathbf c.\ 0.$$
\item Le produit $\prod_{i=1}^n (5a_i)$ est égal à 
$$\mathbf a.\ 5\prod_{i=1}^n a_i\ \ \mathbf b.\ 5^n\prod_{i=1}^n a_i\ \ \mathbf c.\ 5^{n-1}\prod_{i=1}^n a_i.$$
\end{itemize}\newpage\section{Indicators}\section{Du texte aux quantificateurs}pas d'indication :(\section{QCM}\begin{itemize}
\item
\item
\item
\end{itemize}\newpage\section{Corriges}\section{Du texte aux quantificateurs}\begin{itemize}
\item On peut l'écrire sous la forme : 
$\exists C\in\mathbb R$, $\forall x\in\mathbb R$, $f(x)=C$; une autre écriture possible est $\forall x,y\in\mathbb R,\ f(x)=f(y)$.
\item Si on nie l'assertion précédente, on trouve $\forall C\in\mathbb R,$ $ \exists x\in\mathbb R$, $f(x)\neq C$. Si on nie la seconde, on trouve $\exists x,y\in\mathbb R,\ f(x)\neq f(y)$.
\item $\exists x\in\mathbb R,\ f(x)=0$;
\item $\exists T\in\mathbb R^*$, $\forall x\in\mathbb R$, $f(x+T)=f(x)$.
\end{itemize}\section{QCM}\begin{itemize}
\item On somme $(n+1)$ fois le nombre 2. La bonne réponse est b.
\item On somme $(n+1)$ fois le nombre 1 (pour les $p$ correspondant à $0,2,\dots 2n$), et $(n+1)$ faut le nombre $-1$ (pour les $p$ correspondant à $1,3,\dots,2p+1$). La bonne réponse est c. (Si vous n'êtes pas convaincu, essayez le calcul avec $n=2,3,...$).
\item Dans chaque produit, il y a le terme 5 qui ne dépend pas de $i$ et qu'on peut extraire du produit. Comme il y a $n$ termes dans le produit, la bonne réponse est b.
\end{itemize}
\end{document}
